One of the biggest limitations to the accuracy of our model was the limited amount of available preictal data.  There were twenty times more interictal training segments than their were preictal segments, and there were only about 24 preictal segments per subject.  This limited amount of data increases the likelihood that our SVM model will treat the preictal segments as noise and that greatly impacts our prediction accuracy.  Another substantial limitation was that imposed by available compute power.  The dataset for the Kaggle challenge was around 60 GB worth of data. Computation of the $STL_{max}$ algorithm on the full dataset took about ten days of compute time on an octacore compute node with 20GB of available RAM.  Having such long delays before knowing the results of the slightest change is quite preventative of fast progress.

Overall, I was able to achieve an aucROC score of 72\% in the Kaggle competition.  The competition came to a close on November 17th and the winner of the competition achieved an aucROC score of 84\%.  While these scores certainly indicate performance greater than random guessing, there is still a great deal of improvement that can be had.  One of the greatest difficulties in accurately predicting seizures is the lack of obvious feature choices.  To date, there has not been a particular feature that has been found to be a majority principal component of the prediction accuracy.  In future work I am interested in pursuing the effectiveness of convolutional network models and other unsupervised learning techniques.  Unsupervised learning techniques may hold the key to discovering relationships within the EEG that cannot be simply formulated with our knowledge today.
