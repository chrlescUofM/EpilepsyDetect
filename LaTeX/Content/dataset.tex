The dataset we used was provided by the American Epilepsy Society and is hosted through Kaggle.  The full dataset can be downloaded from \href{http://www.kaggle.com/c/seizure-prediction/data}{here}.  The dataset consists of continuous time intercranial EEG recordings from seven different subjects, five dogs and two humans.  The recordings are broken down into ten minute clips labeled as either "preictal" for pre-seizure segments or "interictal" for non-seizure segments.  However, sets of six sequential segments from the training data correspond to adjacent time points, resulting in one hour groupings of continuous time recordings.  For the scope of this project, preictal is defined to cover the hour prior to seizure on-set with a five minute event horizon.  The horizon is enforced to ensure that the resulting algorithm can predict seizures with enough warning to allow for seizure counter measures to be put in place.  Additionally, the horizon ensures that any potentially missed ictal recording is not included in the pre-ictal segments.  Interictal data segments were chosen randomly from the full amount of data, with the enforced restriction that interictal segments be as far away from any seizure as possible.  This resulted in a separation from any seizure activity of one week in the canine recordings and of four hours in the human recordings.  The test data consisted of unlabeled ten minute segments of preictal and interictal recording.
