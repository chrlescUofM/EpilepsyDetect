Epileptic seizures affect approximately one percent of the worlds population \cite{mirowski08}.  Seizures pose a significant health risk to the millions of people they affect.  There is the obvious physical toll they place on the body, however, there is also a severe emotional toll.  Research suggests that for many seizure patients the emotional fear of impending seizures is often one of the biggest concerns, for some even more so than the actual seizures \cite{fisher00}.  Finally, seizure patients also suffer from dangerous situational risk.  For example, operating a motor vehicle and suddenly having a seizure is far more dangerous than suddenly having a seizure while sedentary.  Accurately predicting the on-set of seizures therefore has the ability to greatly improve the quality of life for patients suffering from frequent epileptic seizures.  Warning the patient of an impending seizure would allow them to remove themselves from dangerous situations and emotionally prepare themselves for a possible event.  In the ideal case preventative treatment, such as fast acting medicine or electrical stimulation, may even be capable of avoiding the seizure altogether.


Despite a significant amount of research no methods of seizure prediction have thus far achieved clinical applicability \cite{mirowski08}.  All current methods suffer from the trade-off between sensitivity and specificity, adjusting parameters to improve the detection rate inevitably results in the generation of more false positives.  Too many false positives and a prediction scheme becomes useless, the patient will simply begin to ignore warnings of seizures, even when correct.  We address the seizure prediction problem through the analysis of bivariate feature measures.  Our methods succeed in distinguishing between interictal (between seizure) and preictal (preceding seizure) segments with some degree of success.  However, the algorithms developed here within still face many shortcomings and leave room for the development of future work on-top of our model. 

The organization of this paper is as follows.  Section \ref{sec:data} contains a description of the dataset we used in our analysis, section \ref{sec:methods} contains an overview of our methods, section \ref{sec:results} contains a discussion and analysis of our results, and section \ref{sec:conclusion} addresses our conclusions and directions of future work.
